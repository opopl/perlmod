{
\large
\begin{longtable}[l]{llr}
\caption[\texttt{theme=Zurich, type=longtable, left=1}]{\texttt{theme=Zurich, type=longtable, left=1}. }\\
\toprule
\multicolumn{2}{c}{\textbf{Item}} & \multicolumn{1}{c}{\textbf{}}            \\
\cmidrule(r){1-2}
\textbf{Animal}                   & \multicolumn{1}{c}{\textbf{Description}} & \multicolumn{1}{c}{\textbf{Price}} \\
\midrule
\endfirsthead
\caption[]{Continued from previous page}\\

\toprule
\multicolumn{2}{c}{\textbf{Item}} & \multicolumn{1}{c}{\textbf{}}            \\
\cmidrule(r){1-2}
\textbf{Animal}                   & \multicolumn{1}{c}{\textbf{Description}} & \multicolumn{1}{c}{\textbf{Price}} \\
\midrule
\endhead
\hline
\multicolumn{3}{r}{{Continued on next page}} \\
\bottomrule
\endfoot

\endlastfoot
Gnat      & per gram & 13.65 \\
          & each     & 0.01  \\
Gnu       & stuffed  & 92.59 \\
Emu       & stuffed  & 33.33 \\
Armadillo & frozen   & 8.99  \\
Gnat      & per gram & 13.65 \\
          & each     & 0.01  \\
Gnu       & stuffed  & 92.59 \\
Emu       & stuffed  & 33.33 \\
Armadillo & frozen   & 8.99  \\
Gnat      & per gram & 13.65 \\
          & each     & 0.01  \\
Gnu       & stuffed  & 92.59 \\
Emu       & stuffed  & 33.33 \\
Armadillo & frozen   & 8.99  \\
Gnat      & per gram & 13.65 \\
          & each     & 0.01  \\
Gnu       & stuffed  & 92.59 \\
Emu       & stuffed  & 33.33 \\
Armadillo & frozen   & 8.99  \\
Gnat      & per gram & 13.65 \\
          & each     & 0.01  \\
Gnu       & stuffed  & 92.59 \\
Emu       & stuffed  & 33.33 \\
Armadillo & frozen   & 8.99  \\
Gnat      & per gram & 13.65 \\
          & each     & 0.01  \\
Gnu       & stuffed  & 92.59 \\
Emu       & stuffed  & 33.33 \\
Armadillo & frozen   & 8.99  \\
Gnat      & per gram & 13.65 \\
          & each     & 0.01  \\
Gnu       & stuffed  & 92.59 \\
Emu       & stuffed  & 33.33 \\
Armadillo & frozen   & 8.99  \\
Gnat      & per gram & 13.65 \\
          & each     & 0.01  \\
Gnu       & stuffed  & 92.59 \\
Emu       & stuffed  & 33.33 \\
Armadillo & frozen   & 8.99  \\
Gnat      & per gram & 13.65 \\
          & each     & 0.01  \\
Gnu       & stuffed  & 92.59 \\
Emu       & stuffed  & 33.33 \\
Armadillo & frozen   & 8.99  \\
\bottomrule
\end{longtable}
}
